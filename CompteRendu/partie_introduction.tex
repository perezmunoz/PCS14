\section{Introduction}

\par Depuis 2008, un logiciel est apparu sur internet, suscitant un certain engouement général. En effet, avec des quantités de données croissant de manière exponentielle, les entreprises s'intéressent aux solutions permettant leur analyse, de manière distribuée, afin de pouvoir effectuer des analyses en un temps raisonnable. Ces logiciels existaient déjà, mais supposaient un matériel d'assez bonne facture, car peu résistants à la panne. Hadoop arrive donc sur le marché en proposant un système de fichiers distribué avec une haute tolérance à la panne, permettant d'employer des machines de type \og low cost\fg{}. Cela dit, cette solution libre proposée par \emph{Apache} a subi de nombreux changements depuis le début, rendant les documentations spécifiques à chaque version, et les problèmes d'installations très fréquents. La popularité d'Hadoop est donc à la fois liée à son système de fichiers distribué très performant (HDFS), mais également à la difficulté de sa configuration et de son installation.
\par Ayant déjà travaillé sur une interface graphique permettant de se connecter en SSH sur un serveur, et d'effectuer une commande de noeuds à l'aide du logiciel de gestion de ressources \texttt{OAR}, nous nous proposons, dans le cadre de ce projet, d'étudier Hadoop de manière générale, à travers ses différentes versions, et finalement de l'installer sur le cluster Skynet de Supélec. Pour ce faire, on étudiera dans un premier le composant principal, à savoir HDFS, en précisant ses avantages, et les concepts fondamentaux (sur lesquels s'appuient également les systèmes de fichiers distribués en général). Dans un second temps, on expliquera de manière générale la principale application d'Hadoop, à savoir le modèle de programmation \emph{MapReduce} popularisé par Google. Nous mettrons ensuite en application nos connaissances afin d'installer Hadoop, en SingleNode dans un premier temps avec la version \texttt{1.2.1}, puis en MultiNodes sur le cluster Skynet de Supélec avec la version \texttt{2.2.0} pour terminer.

%%% Local Variables: 
%%% mode: latex
%%% TeX-master: "CompteRendu"
%%% End: 
