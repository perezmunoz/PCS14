\section{Hadoop Distritubted File System}

\par HDFS pour \textit{Hadoop Distributed File System} est le système de fichiers propre à Hadoop. C'est le composant en charge du stockage des données dans un cluster Hadoop.

\par Alex je te laisse le soin de détailler HDFS...

\subsection{Fonctionnement}

\par Le fonctionnement de HDFS s'appuie sur plusieurs démons :

\begin{itemize}
\item le NameNode (NN) : c'est le noeud maître disposant d'une machine dédiée;
\item le SecondaryNameNode (SNN) : tout comme le NN c'est aussi un noeud maître disposant d'une machine dédiée. Ce noeud vient comme son nom l'indique seconder le NN en cas de panne majeure afin de ne pas perdre l'arborescense des données stockées dans le HDFS;
\item le DataNode (DN) : c'est un noeud esclave contenant les données du HDFS et implanté sur chaque machine du cluster.
\end{itemize}

\par A titre d'exemple, dans un cluster de 50 machines, vous trouverez trois noeuds maîtres correspondant au NameNode, SecondaryNameNode et JobTracker (confère partie suivante sur MapReduce). Il ne reste plus que 47 noeuds esclaves contenant chacun une copie du DataNode et du TaskTracker (confère partie suivante sur MapReduce). Détaillons les noeuds maîtres.

\subsubsection{Le NameNode}

\par à compléter

\subsubsection{Le SecondaryNameNode}

\par à compléter

%%% Local Variables: 
%%% mode: latex
%%% TeX-master: "CompteRendu"
%%% End: 
