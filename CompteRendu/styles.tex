\documentclass[12pt,a4paper]{article}
\usepackage[utf8]{inputenc}
\usepackage[T1]{fontenc} 
\usepackage[french]{babel}
\usepackage{fullpage}
\usepackage{amsmath}
\usepackage{amsfonts}
\usepackage{amssymb}
\usepackage{color}
\usepackage{titlesec}
\usepackage[french]{varioref}
\usepackage{graphicx}
\usepackage{subfig}

% Aerer les tableaux
\renewcommand{\arraystretch}{1.5}
\setlength{\tabcolsep}{0.5cm}

\makeatletter
\definecolor{vert}{rgb}{0,0.5,0}
\definecolor{vertclair}{rgb}{0.85,1,0.85}
\definecolor{redclair}{rgb}{1,0.9,0.75}

\titleformat*{\section}{\color{red}\scshape\bfseries\LARGE}
\titleformat*{\subsection}{\scshape\bfseries\large}
\titleformat*{\subsubsection}{\normalsize\bfseries}

\titlespacing*{\section}{0cm}{0.5cm}{0.5cm}
\titlespacing*{\subsection}{1cm}{0.5cm}{0.3cm}
\titlespacing*{\subsubsection}{1.75cm}{0.3cm}{0.3cm}

\renewcommand{\thesection}{\textcolor{red}{\Roman{section}}}
\renewcommand{\thesubsection}{\textcolor{blue}{\Alph{subsection}}}
\renewcommand{\thesubsubsection}{\textcolor{vert}{\arabic{subsubsection}}}

\newcommand{\mnp}[1]{\mathcal{M}_{n,p}\left(\bb{#1}\right)}

% Le vocabulaire
\newcommand{\vocabulaire}[1]{\\ \begin{minipage}{1\textwidth}\textbf{Vocabulaire :} \begin{center}\begin{minipage}{0.8\textwidth}
        #1
      \end{minipage}\end{center}\end{minipage}}

% les théorèmes
\newtheorem{thm}{\textsc{Théorème}}
\newcommand{\theoreme}[2]{
  \begin{center}
    \fcolorbox{red}{redclair}{
      \begin{minipage}{0.9\textwidth}
        \begin{thm}[\emph{#1}] \color{red} \ \newline
              #2
        \end{thm}
      \end{minipage}}
  \end{center}}

% Résultat important
\newcommand{\important}[1]{
  \vskip 0.5cm
  \fcolorbox{red}{white}{
    \begin{minipage}[c]{0.9\linewidth}
      #1
    \end{minipage}
  }
  \vskip 0.5cm
}

% Les remarques
\newtheorem{rmq}{Remarque}
\newcommand{\remarque}[1]{
  \fcolorbox{black}{white}{
    \begin{minipage}{1\textwidth}
      \begin{rmq}
        #1
      \end{rmq}
    \end{minipage}}\smallskip}

% Les lemmes
\newtheorem{lem}{\textsc{Lemme}}
\newcommand{\lemme}[2]{
  \begin{center}
    \fcolorbox{red}{white}{
      \begin{minipage}{0.9\textwidth}
        \begin{lem}
          \label{#1}
          \textcolor{red}{#2}
        \end{lem}
      \end{minipage}}
  \end{center}\smallskip}

% Les preuves
\newcounter{pr}
\newcommand{\preuve}[3]{
  \addtocounter{pr}{1}
  \begin{center}
    \begin{minipage}{0.6\textwidth}
      \begin{center}
        \textbf{Démonstration \thepr \enspace-- #2 \ref{#1}}
      \end{center}
      #3
    \end{minipage}
  \end{center}\smallskip}

% cas 1,2...
\newlength{\caslength}
\newcommand{\cas}[2]{
  \begin{flushright}
    \begin{minipage}{0.9\textwidth}
      \settowidth{\caslength}{Cas #1 : }
      \parbox[b]{\caslength}{\textbf{Cas #1 :}}
      \parbox[t]{0.8\textwidth}{#2}
    \end{minipage}
  \end{flushright}
}

% méthode
\newsavebox{\methodebox}
\newenvironment{methode}[1]
{\small\savebox{\methodebox}{\textsc{Méthode : #1 }}\usebox{\methodebox}\\
  \begin{tabular}{rl}}
  {\normalsize\end{tabular}}

\newtheorem{exempletheoreme}{\textbf{Exemple}}
\newenvironment{exemple}{
  \begin{exempletheoreme}\normalfont \
    \begin{center}
      \begin{minipage}{0.9\textwidth}}
      {\end{minipage}
    \end{center}
  \end{exempletheoreme}
}


% les définitions
\newtheorem{definitiontheoreme}{\textbf{Définition}}
\newcommand{\definition}[2]{
  \begin{center}
    \fcolorbox{vert}{vertclair}{
      \begin{minipage}{1\textwidth}
        \begin{definitiontheoreme} \normalfont (\emph{#1}) \color{vert} \newline
          #2
        \end{definitiontheoreme}
      \end{minipage}}
  \end{center}}
